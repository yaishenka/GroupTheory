\documentclass{article}

\usepackage[a4paper]{geometry}
\usepackage{mathtools,amssymb}
\usepackage{array}
\newcolumntype{P}[1]{>{\centering\arraybackslash}p{#1}}

\usepackage[T1,T2A]{fontenc}
\usepackage[utf8]{inputenc}
\usepackage[russian]{babel}
\usepackage{multirow}
\usepackage{hhline}
\usepackage{cancel}

\usepackage[useregional]{datetime2}
\usepackage[thinlines]{easytable}
\usepackage{cancel}

\title{Теория групп. Лекция 15}
\author{Штепин Вадим Владимирович}
\date{\DTMdate{2019-12-12}}

\begin{document}
\maketitle

\section{Периодическая часть группы}

Пусть $G$ "--- конечнопорожденная абелева группа относительно сложения.

\underline{Опр.} \textbf{Периодическая часть группы $G$ (кручение $G$)} $Tor(G) = \{a \in G \mid ord(a) < \infty\}$.

\vspace{5pt}

\textbf{Утв.}

$Tor(G) \leq G$

\vspace{5pt}

\textbf{Доказательство}

Пусть $a_1, a_2 \in Tor(G)$. Тогда $\exists n_1, n_2 \ n_1a_1 = n_2a_2 = 0$. Очевидно, что $n_1n_2(a_1 + a_2) = 0 \Rightarrow a_1 + a_2 \in Tor(G)$. Более того, $ord(a_1 + a_2) | ord(a_1)ord(a_2)$. Аналогично, $-a_1 \in Tor(G)$.

\vspace{5pt}

\textbf{Утв.}

Пусть $G = H \oplus Z^k$, где $H$ "--- прямая сумма примарных циклических подгрупп $H = Z_{p_1^{\alpha_1}} \oplus ... \oplus Z_{p_s^{\alpha_s}}$. Тогда $Tor(G) = H$ и $G / Tor(G) \simeq Z^k$

\vspace{5pt}

\textbf{Доказательство}
Так как все элементы $H$ имеют конечный порядок, то $H \leq Tor(G)$. Покажем, что $Tor(G) \leq H$. Пусть $a \in Tor(G) \Rightarrow \exists n \ na = 0$ и $n(a + H) = H \Rightarrow a + H \in Tor(G / H) \Rightarrow Tor(G) / H \leq Tor(G / H)$. Так как $G / H \simeq Z^k$, то $Tor(G / H) = \{e\}$ (в $Z^k$ нет элементов конечного порядка). Значит, $Tor(G) / H = H$ и $Tor(G) + H = H$. По критерию принадлежности смежному классу, $Tor(G) \leq H$.

Тогда $Tor(G) = H$ и $G / Tor(G) = Z^k$

\vspace{10pt}

\textbf{Замечание}

В разложении конечнопорожденной абелевой группы $G$ в прямую сумму циклических прямая сумма примарных циклических подгрупп и слагаемое $Z^k$ определены однозначно с точностью до изоморфизма.

В частности, $k = rk(Z^k)$ определено однозначно.

Однако то, что разложение единственно пока еще не доказано.

\vspace{5pt}

\underline{Опр.} \textbf{$p$-кручение} конечнопорожденной абелевой группы $H$ "--- это множество $Tor_p(H) = \{a \in H \mid ord(a) = p^l$ для некоторого $\ l\}$, где $p$ "--- простое число.

\vspace{5pt}

\textbf{Утв.}

Пусть $H = Z_{p_1^{\alpha_1}} \oplus ... \oplus Z_{p_s^{\alpha_s}}$ и все $p_i$ простые (возможно, совпадающие). Тогда $Tor_p(H) \leq H$ и $Tor_p(H) = \oplus Z_{p_i^{\alpha_i}}$ "--- сумма по всем слагаемым, для которых $p_i = p$

\vspace{5pt}

\textbf{Доказательство}

$a_1, a_2 \in Tor_p(H) \Rightarrow \exists k_1, k_2 \in N$, что $p^{k_1}a_1 = p^{k_2}a_2 \Rightarrow ord(a_1 + a_2) | p^{k_1 + k_2}$ и $a_1 + a_2 \in Tor_p(H)$

Пусть $a \in H$, $a = (a_1, ..., a_s),\ a_i \in 	Z_{p_i^{\alpha_i}}$. Если $ord(a) = $ НОК$(ord(a_1), ..., ord(a_s))$ "--- степень числа $p$, то $ord(a_i) = 1$ для всех $i$, что $p_i \neq p$.

Значит, $Tor_p(H) = \oplus Z_{p_i^{\alpha_i}}$ по всем $i$, что $p_i = p$.

\vspace{10pt}

\textbf{Следствие}

В разложении конечнопорожденной абелевой группы в прямую сумму примарных циклических подгрупп сумма слагаемых, относящихся к одному простому числу $p$ определяется однозначно, так как она есть $p$-кручение, независимое от разложения.

Осталось доказать, что если абелева группа разлагается в прямую сумму примарных циклических, относящихся к одному простому числу $p$, то все слагаемые определены однозначно.

\vspace{5pt}

\textbf{Утв.}

Пусть $H = Z_p^{\alpha_1} \oplus ... \oplus Z_p^{\alpha_s}$, $p$ "--- простое. Тогда порядкислагаемых однозначно восстанавливаются по группе $H$ с точностью до перестановки слагаемых.

\vspace{5pt}

\textbf{Доказательство}

Индукция по порядку группы $H$.
\begin{enumerate}
	\item База: $H = \{e\}$ "--- верно, так как разложение пустое.
	\item Переход: Пусть верно для всех групп порядка меньше, чем $p^l$. Тогда $pH = \{ph \mid h \in H\} \leq H$ и $pH = pZ_p^{\alpha_1} \oplus ... \oplus pZ_p^{\alpha_s} \simeq Z_p^{\alpha_1-1} \oplus ... \oplus Z_p^{\alpha_s-1}$. Так как $H / pH = Z_p \oplus ... \oplus Z_p$ с тем же количеством слагаемых, то число слагаемых в исходной сумме определено однозначно.
	
	Применим предположение индукции к $pH$ и получим, что набор ненулевых показателей определен однозначно (они же равны $\alpha_i-1$.
	
	Но в силу того, что общее число слагаемых определено однозначно, то и число показателей, в которых $\alpha_i - 1 = 0$ так же определено однозначно.
	
	Таким образом, мы однозначно определили весь набор $\alpha_i$.
\end{enumerate}

\vspace{10pt}

\textbf{Теорема (о единственности разложения конечнопорожденной абелевой группы в прямую сумму циклических}

Пусть $G$ "--- конечнопорожденная абелева группа. Тогда $G$ допускает разложение $G = Z_{p_1^{\alpha_1}} \oplus ... \oplus  Z_{p_s^{\alpha_s}} \oplus Z^k$, где $p_i$ "--- простые (возможно, совпадающие) числа, а $s \in N$, возможно, нулевое.

Причем разложение единственно с точностью до перестановки слагаемых

\vspace{10pt}

\textbf{Следствие}

Если $G$ "--- конечнопорожденная абелева группа и $Tor(G) = \{e\}$ ранга $k$, то $G \simeq Z^k$ для единственного $k$.

Так же, при $k \neq l \ Z^k \neq Z_l$.

\vspace{10pt}

\textbf{Теорема (об описании конечных подгрупп в мультипликативной группе поля}

Пусть $F$ "--- поле, $F^* = F \setminus \{0\}$ "--- мультипликативная группа поля. Тогда, если $G \leq F^*$ и $|G| \leq \infty$, то $G$ "--- циклическая.

\vspace{5pt}

\textbf{Доказательство}

Очевидно, что $G$ абелева и конечнопорождена. Тогда, $G = Z_{u_1} \times ... \times Z_{u_k}$, $u_i \geq 1$ и $u_1 | u_2 | ... | u_k$. $\forall a \in G$ верно $a^{u_k} = 1$, так как $u_k \ \vdots \ u_i \ \forall i$. Значит, $a$ "--- корень многочлена $x^{u_k} - 1$, а число корней многочлена не больше его степени, а значит $|G| \leq u_k$ и $|G| = u_1...u_k$, а значит $G \simeq Z_{u_k}$.

\vspace{5pt}

\textbf{Следствие}

Для конечного поля верно, что $F^*$ "--- циклическая группа порядка $|F| - 1$.

\section{Кольца и алгебры} 

\underline{Опр.} \textbf{Кольцо} "--- множество $R$ с определенными операциями сложения и умножения, где $(R, +)$ "--- абелева группа, $(R, *)$ "--- полугруппа (требуется только ассоциативность) и верна дистрибутивность $c(a + b) = ca + cb$, $(a + b)c = ac + bc \ \forall a, b, c \in R$.

Если умножение коммутативно, то $R$ "--- \textbf{коммутативное} кольцо (не абелево, так как этот термин применим только к группам).

Если в $R$ есть нейтральный по умножению элемент, то он называется \textbf{единицей} и обозначается $1$.

\vspace{5pt}

\underline{Опр.} \textbf{Алгебра} "--- кольцо, являющееся линейным пространством над некоторым полем $F$ и верно $\forall \alpha \in F \ \forall a, b \in R \ \alpha(ab) = (\alpha a)b = a(\alpha b)$

\vspace{10pt}

\textbf{Примеры:}
\begin{enumerate}
	\item Если $F$ "--- поле, то $F$ "--- алгебра размерности 1 над самим собой.
	\item $(Z, +, *)$ "--- кольцо
	\item $F[x]$ "--- алгебра многочленов над полем $F$.
	\item $(M_{n \times n}(F), +, *)$ "--- алгебра матриц над полем $F$.
\end{enumerate}


Обе алгебры 3, 4 являются кольцами с единицей

\vspace{5pt}

\underline{Опр.} \textbf{Подкольцо}(\textbf{подалгебра}) "--- непустое подмножество кольца (алгебры), само являющееся кольцом (алгеброй) относительно операций, определенных на исходной структуре.

\vspace{10pt}

\textbf{Теорема (критерий подкольца)}

$K \subset R$ "--- подкольцо $\Leftrightarrow K$ замкнуто относительно умножения и вычитания (сложения с обратным элементом по сложению).

\vspace{5pt}

\underline{Опр.} Пусть $R, S$ "--- кольца, тогда $\phi: R \rightarrow S$ "--- \textbf{гомоморфизм} колец (колец с единицей), если $\phi$ сохраняет операции сложения и умножения (для колец с единицей следует требовать $\phi(1_R) = 1_S$, так как это не следует из определения гомоморфизма колец)

$Im(\phi) = \{\phi(a) \mid a \in R\}$

$Ker(\phi) = \{a \in R \mid \phi(a) = 0_S\}$

\vspace{5pt}

\textbf{Примеры:}
\begin{enumerate}
	\item $\phi: Z \rightarrow Z_n, \ \phi(a) = a + nZ$ "--- гомоморфизм колец с единицей, так как $\phi(1) = 1 + nZ = 1_{Z_n}$
	\item $R$ "--- алгебра матриц $M_{n \times n}(F)$, $S$ "--- алгебра матриц $M_{k \times k}(F)$, $\phi(A) = \begin{pmatrix} 
	A & 0 \\
	0 & 0 \end{pmatrix}$ "--- гомоморфизм алгебр (без единицы, так как $\phi(1_R) \neq 1_S$.
\end{enumerate} 

\vspace{5pt}

\textbf{Опр.} Пусть $R$ "--- кольцо. Подкольцо $I$ "--- \textbf{левый (правый) идеал} в $R$, если $\forall x \in R \ \forall a \in I$ верно $xa \in I$ ($ax \in I$), то есть левый (правый) идеал выдерживает умножение слева (справа) на элементы кольца.

Если идеал двусторонний, то он называется просто идеалом

\vspace{5pt}

\textbf{Утв.}

Пусть $\phi: R \rightarrow S$ "--- гомоморфизм колец. Тогда $Im(\phi) \leq S$ (подкольцо) и $Ker(\phi) \triangleleft R$ (идеал)

\vspace{5pt}

\textbf{Доказательство}

$\phi: (R, +) \rightarrow (S, +)$ "--- гомоморфизм групп, значит $Im(\phi) \leq S$ "--- подгруппа. Проверим замкнутость относительно умножения: $\phi(a)\phi(b) = \phi(ab) \in Im(\phi)$.

Известно, что $Ker(\phi) \triangleleft (R, +)$ "--- нормальная подгруппа. Пусть $a \in Ker(\phi)$, $x \in R$. Тогда $\phi(ax) = \phi(xa) = 0$, так как $\phi(a) = 0$, то есть $ax, xa \in Ker(\phi)$ и $Ker(\phi)$ "--- идеал

\section{Факторкольцо по идеалу}

Пусть $R$ "--- кольцо и $I$ "--- его идеал.

\underline{Опр.} \textbf{Факторкольцо по идеалу} "--- факторгруппа $(R, +)$ по $(I, +)$ $R / I = \{x + I \mid x \in R\}$ с определенной операцией умножения $(x + I)(y + I) = xy + I \in R / I$.

Данное определение корректно. Пусть $x + I = x' + I$ и $y + I = y' + I$. Тогда $x' = x + a$ и $y' = y + b$, $a, b \in I$ и $(x' + I)(y' + I) = x'y' + I = (x + a)(y + b) + I = xy + ay + xb + ab + I = xy + I$, так как $I$ "--- идеал.

\vspace{10pt}

\textbf{Теорема}

Множество смежных классов по идеалу $I$ "--- кольцо относительно операций $(x + I) + (y + I) = (x + y) + I$ и определенной выше операции умножения.

Отображение $p: R \rightarrow R / I \ p(x) = x + I$ "--- эпиморфизм (сюръективный гомоморфизм) колец.

\vspace{5pt}

\textbf{Доказательство}

$p(x + y) = x + y + I = p(x) + p(y)$

$p(xy) = xy + I = (x + I)(y + I) = p(x)p(y)$

$p$ сюръективен для групп, а значит и для колец. Образ кольца при гомоморфизме есть кольцо, а значит $R / I$ "--- кольцо

\vspace{10pt}

\underline{Опр.} Построенное кольцо "--- \textbf{факторкольцо} $R$ по идеалу $I$. $p$ "--- \textbf{канонический эпиморфизм}.

\vspace{10pt}

\textbf{Теорема (о гомоморфизмах колец)}

Пусть $\phi: R \rightarrow S$ "--- гомоморфизм колец. Тогда $Im(\phi) \leq S$ и $Ker(\phi) \triangleleft R$ и $Im(\phi) \simeq R / Ker(\phi)$, причем существует такой изоморфизм $\psi: Im(\phi) \rightarrow R / Ker(\phi)$, что $\psi \circ \phi = p$, где $p$ "--- канонический эпиморфизм.

\vspace{5pt}

\textbf{Доказательство}

$\phi: (R, +) \rightarrow (S, +)$ "--- гомоморфизм групп. Для соответствующих групп верна теорема о гомоморфизме групп: $Im(\phi) \simeq R / Ker(\phi)$. Обозначим этот изоморфизм $\psi: Im(\phi) \rightarrow R / Ker(\phi)$. Этот гомоморфизм групп сохраняет операцию умножения, так как по определению $\psi(x)$ есть полный прообраз $x$, а это есть смежный класс $a + Ker(\phi)$, где $a$ "--- произвольный элемент $R$, что $\phi(a) = x$. 

Тогда $\psi(xy) = (ab + Ker(\phi) = (a + Ker(\phi))(b + Ker(\phi)) = \psi(x)\psi(y)$.
\end{document}