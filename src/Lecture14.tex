\documentclass{article}

\usepackage[a4paper]{geometry}
\usepackage{mathtools,amssymb}
\usepackage{array}
\newcolumntype{P}[1]{>{\centering\arraybackslash}p{#1}}

\usepackage[T1,T2A]{fontenc}
\usepackage[utf8]{inputenc}
\usepackage[russian]{babel}
\usepackage{multirow}
\usepackage{hhline}
\usepackage{cancel}

\usepackage[useregional]{datetime2}
\usepackage[thinlines]{easytable}
\usepackage{cancel}

\title{Теория групп. Лекция 14}
\author{Штепин Вадим Владимирович}
\date{\DTMdate{2019-12-05}}

\begin{document}
\maketitle

\section{Свободные абелевы группы}

\underline{Опр.} Абелева группа $C_1 \times ... \times C_k$ "--- \textbf{конечнопорожденная}, если $\forall i \  C_i$ "--- циклическая (возможно, бесконечного порядка). 

В дальнейшем будем считать операцию сложением.

\vspace{5pt}

\underline{Опр.} Пусть $G$ "--- конечнопорожденная абелева группа. Система элементов $A = \{a_1, ..., a_n\}$ \textbf{независима}, если из условия $\sum \limits_i \lambda_ia_i = 0$ следует, что все $\lambda_i = 0$.

\vspace{5pt}

\underline{Опр.} Система элементов $A = \{a_1, ..., a_n\}$ "--- \textbf{базис} в $G$, если это независимая система и $G = \langle a_1, ..., a_n \rangle$.

\vspace{10pt}

\textbf{Замечание}

Если $G$ "--- конечнопорожденная абелева группа и $e_1, ..., e_n$ "--- базис, то каждый элемент однозначно раскладывается по базису.

\vspace{10pt}

\textbf{Замечание}

Не во всякой конечнопорожденной абелевой группе есть базис.

\textbf{Пример}

$Z_n = Z / nZ$ "--- конечнопорождена элементом 1, но она не обладает базисом, так как $\forall a \in Z_n \ na = 0$ и любая система зависима.

\vspace{5pt}

\underline{Опр.} Группа $A$ (абелева, конечнопорожденная) "--- \textbf{свободная абелева группа} ранга $n$, если в ней существует базис из $n$ элементов.

\vspace{10pt}

\textbf{Утв.}

Всякая свободная абелева группа ранга $n$ изоморфна $Z^n$.

\vspace{5pt}

\textbf{Доказательство}

Пусть $e_1, ..., e_n$ "--- базис $A$. Тогда каждому элементу $a \in A$ однозначно сопоставляется столбец его координат в базисе. Это соотвествие линейно, а значит это гомоморфизм групп. Биекция следует из однозначности разложения по базису.

\vspace{10pt}

\textbf{Теорема}

Любые два базиса свободной абелевой группы равномощны.

\vspace{5pt}

\textbf{Доказательство}

Пусть $e_1, ... , e_n$, $f_1, ..., f_k$ "--- базисы и $k > n$. Тогда $(f_1, ..., f_k) = (e_1, ..., e_n)S$, где $S$ "--- матрица перехода между базисами (получена разложением $f_i$ по базису $e_1, ..., e_n$). $S \in M_{n \times k}(Z) \subset M_{n \times k}(Q)$. 

СЛУ $Sx = 0$ (над $Q$) из $n$ уравнений с $k$ неизвестными при $n < k$ обязательно имеет нетривиальное решение $x_0$. Умножая, при необходимости, на НОК всех знаменателей координат, можно считать, что решение целочисленно. Значит, $(f_1, ..., f_k)x_0 = (e_1, ..., e_n)Sx_0 = 0$ и $f_1,..., f_k$ "--- не базис.

\section{Строение конечнопорожденной абелевой группы}

\textbf{Теорема}

Пусть $G$ "--- конечнопорожденная свободная абелева группа с базисом $e = (e_1, ..., e_n)$. Тогда $f = (f_1, ..., f_n)$ "--- базис в $G \Leftrightarrow (f_1, ..., f_n) = (e_1, ..., e_n)S$, где $S$ "--- матрица перехода и $det(S) \in \{1, -1\}$.

\vspace{5pt}

\textbf{Доказательство}
\begin{enumerate}
	\item Необходимость.
	
	$(e_1, ..., e_n) = (f_1, ..., f_n)T$ "--- разложение $e$ по базису $f$, где $T$ "--- матрица перехода. Тогда $(e_1, ..., e_n) = (e_1, ..., e_n)ST$. В силу единственности разложения, $ST = E$, а все коэффициенты разложения целые. Значит, $S = T^{-1}$ и $det(S) = det(T) \in \{1, -1\}$, так как $det(S)det(T) = 1$ и значения определителей "--- целые числа.
	
	\item Достаточность. Пусть $f = eS$ и $det(S) \in \{1, -1\}$. Покажем, что $f$ "--- базис. Очевидно, что существует $S^{-1}$ с целыми коэффициентами, так как $|det(S)| = 1$. Значит, $(e_1, ..., e_n) = (f_1, ..., f_n)S^{-1}$ и $f_1, ..., f_n$ так же порождают $G$. Покажем независимость $f_1, ..., f_n$. Пусть не так и $(f_1, ..., f_n)\begin{pmatrix}
	\alpha_1\\
	.\\
	.\\
	\alpha_n\\
	\end{pmatrix} = 0$. Тогда $(e_1, ..., e_n)S\begin{pmatrix}
	\alpha_1\\
	.\\
	.\\
	\alpha_n\\
	\end{pmatrix} = 0$ и $S\begin{pmatrix}
	\alpha_1\\
	.\\
	.\\
	\alpha_n\\
	\end{pmatrix} = 0$.

Но в силу невырожденности $S$ получаем, что $\begin{pmatrix}
	\alpha_1\\
	.\\
	.\\
	\alpha_n\\
	\end{pmatrix} = 0$.
	
	Значит, система $f_1, ..., f_n$ независима.	
\end{enumerate}

\vspace{5pt}

\textbf{Замечание}

Множество целочисленных матриц с определителем из множества $\{1, -1\}$ образуют группу $GL_n(Z)$. В частности, в $GL_n(Z)$ содержатся элементарные матрицы:
\begin{enumerate}
	\item $E + tE_{i,j}, \ i \neq j, t \in Z$ "--- матрицы, в которых на главной диагонали стоят единицы, и некоторое число вне главной диагонали равно $t$.
	\item $diag(\pm1, ... , \pm1)$
	\item Единичная матрица, получаемая из диагональной перестановкой двух строк (столбцов, что эквивалентно).
\end{enumerate}

\vspace{5pt}

\underline{Опр.} Рассмотренные матрицы "--- \textbf{целочисленные элементарные матрицы}, а соответствующие им преобразования "--- \textbf{целочисленные элементарные преобразования}.

\vspace{10pt}

\textbf{Теорема (о подгруппах свободной абелевой группы}

Пусть $G$ "--- САГ, $rk(G) = n$, $H \leq G$. Тогда $H$ "--- САГ и $rk(H) \leq n$.

В качестве свободных абелевых групп ранга ноль будем рассматривать группы, состоящие только из нейтрального элемента.

\textbf{Доказательство}

Индукция по $n$.
\begin{enumerate}
	\item База: если $n = 0$, то $G = H = \{e\}$
	\item Переход: пусть для всех групп $G$, что $rk(G) < n$ верно и $rk(G) = n$, $e_1, ..., e_n$ "--- базис в $G$, $H \leq G$
	
	Пусть $G_1 = \langle e_1, ..., e_{n-1} \rangle$. Тогда $rk(G_1) = n - 1$ и $H_1 = H \cap G_1$. Очевидно, что $H_1 \leq G_1$, и, по предположению индукции, $rk(H_1) \leq n - 1$. Пусть $h_1, ..., h_k$ "--- базис в $H_1$ ($k \leq n-1$). Если $H_1 = H$, то утверждение верно.
	
	Иначе $\exists h \in H \setminus H_1$. Тогда $h = \sum \limits_i e_i\alpha_i$, причем $\alpha_n \neq 0$, так как иначе $h \in H_1$.
	
	Положим $h_{k+1} \in H \setminus H_1$, такой, что $\alpha_n$ минимально возможно ($> 0$). Покажем, что $h_1, ..., h_k, h_{k+1}$ "--- базис $H$.
	
	Пусть $x \in H \setminus H_1$ "--- произвольный. Тогда $x = \sum \limits_i e_i\beta_i$. $\beta_n = q\alpha_n + r$ "--- деление с остатком. Если $r \neq 0$, то $x - qh_{k+1} \in H \setminus H_1$ и его последняя координата в разложении по базису $e_1, ..., e_n$ равна $r > 0$ и $r < \alpha_n$ "--- получаем противоречие с выбором $h_{k+1}$. Значит $r = 0$ и $\beta_n = q\alpha_n$. Тогда $x - qh_{k+1} \in H_1$ (так как $r = 0$). Тогда имеет место представление $x = \sum \limits_i\alpha_ih_i + qh_{k+1}$ и $H = \langle h_1, ..., h_{k+1} \rangle$, так как все $x \in H_1$ разлагаются по $h_1, ..., h_k$. Покажем независимость $h_1, ..., h_{k+1}$.
	
Пусть $\sum_{i = 1}^k\gamma_ih_i + \gamma_{k+1}h_{k+1} = 0$. Если $\alpha_{k+1} = 0$, то в силу независимости $h_1, ..., h_k$ получаем, что $\forall i \  \alpha_i = 0$. Если $\alpha_{k+1} \neq 0$, тогда $\sum_{i = 1}^{k+1}\gamma_ih_i$ имеет ненулевую последнюю координату в разложении по базису $e_1, ..., e_n$ и не может равняться нулю. Значит, $h_1, ..., h_{k+1}$ "--- базис $H$ и $k+1 \leq n$.

\end{enumerate}

\vspace{10pt}

\textbf{Замечание}

Если $H \leq G$ и $G, H$ "--- САГ одного ранга, то не обязательно $H = G$. Пример: $G = Z$, $H = 2Z$ (обе группы ранга 1).

\vspace{5pt}

\textbf{Замечание}

Из того, что элементы независимы не следует, что один из них выражается через остальные. Пример: $2a + 5b + 7c = 0$, но ни один из $a, b, c$ невыразим через другие, так как нельзя делить.

\vspace{10pt}

\textbf{Лемма (о смитовой нормальной форме)}

Пусть $M \in M_{n \times k}(Z)$ и ненулевая. Тогда $\exists P, D, Q: \ M = PDQ$ и $P \in GL_n(Z), \ Q \in GL_k(Z)$, а $D \in M_{n \times k}(Z)$ такая диагональная матрица, что $D_{1,1} \geq 0$, $D_{i,i} | D_{i+1,i+1}$ и, начиная с некоторого $i$ все $D_{i,i} = 0$.

\textbf{Доказательство}

Индукция по $n$
\begin{enumerate}
	\item База: $(a, b) \rightarrow ($НОД$(a, b), 0)$ можно привести алгоритмом Эвклида. Аналогично, $(a_1, ..., a_k) \rightarrow ($НОД$(a_1, ..., a_k), 0, ..., 0)$ можно привести применением $k-1$ раз алгоритма Эвклида, так как НОД$(a_1, ..., a_k) = $НОД(НОД$(a_1, ..., a_{k-1}), a_k)$
	\item Переход: пусть утверджение верно для всех матриц $M$, имеющих меньше $n$ строк. С помощью целочисленных преобразований приведем матрицу к виду, в котором элемент $M_{1,1} = $НОД$(M_{i, j})$, а остальные элементы первой строки и первого столбца равны нулю.
	
	Это можно сделать следующим алгоритмом.
	
	Перенесем минимальный по модулю элемент матрицы в левый верхний угол и начнем занулять первую строку и первый столбец. Если в процессе появится элемент, меньший по модулю, то перенесем его в угол и продолжим.
	
	Если после этого в матрице есть элемент, не делящийся на $M_{1,1}$, то прибавим строку, в которой он находится к первой и продолжим процесс, тем самым получив в углу НОД$(M_{1,1}, *, ..., *)$ "--- НОД элемента в углу и $i$-той строки, меньший, чем $M_{1,1}$. В итоге получим, что все элементы матрицы, получаемой вычеркиванием первой строки и первого столбца делятся на $M_{1,1}$. Приведем ее к смитовой нормальной форме по индукции. 
	
	Так как мы делали элементарные преобразования строк и столбцов, то $D = P_1MQ_1$, $P_1 \in GL_n(Z), \ Q \in GL_k(Z)$ и $M = P_1^{-1}DQ_1^{-1}$
\end{enumerate}

\vspace{5pt}

\textbf{Замечание}

Смитова нормальная форма определена однозначно. Матрицы $P, Q$ определены неоднозначно.

\vspace{5pt}

\textbf{Упражнение}

$u_1...u_t = $НОД$(M_1, ..., M_t)$ "--- однозначно определены, где $M_i$ "--- миноры. В частности, $u_1 = $НОД$(M_1) = $НОД$(M)$.

\vspace{10pt}

\textbf{Теорема (о существовании согласованных базисов в САГ $G$ и $H \leq G$)}

Пусть $G$ "--- САГ ранга $n$, $H \leq G$ ранга $k \leq n$. Тогда в $G$ и $H$ существуют базисы $g_1, ..., g_n$ и $h_1, ..., h_k$, что $h_i = u_ig_i$, где $u_i \in N$ и $u_1 | u_2 | ... | u_k$

\vspace{5pt}

\textbf{Доказательство}

Пусть $e_1, ... e_n$ "--- базис в $G$, $f_1, ..., f_k$ "--- базис в $H$ и оба базиса произвольны. Тогда $(f_1, ..., f_k) = (e_1, ..., e_n)M$, $M \in M_{n \times k}(Z)$. По лемме, $M = PDQ$, где $D$ "--- матрица в смитовой нормальной форме. 

Тогда $(f_1, ..., f_k) = (e_1, ..., e_n)PDQ$ и $(f_1, ..., f_k)Q^{-1} = (e_1, ..., e_n)PD$. Обозначим $(h_1, ..., h_k) = (f_1, ..., f_k)Q^{-1}$ и $(g_1, ..., g_n) = (e_1, ..., e_n)P$ и получим требуемое.

\vspace{10pt}

\textbf{Следствие (о существовании разложения конечнопорожденной абелевой группы в прямую сумму циклических}

Пусть $A$ "--- конечнопорожденная абелева группа. Тогда $A \simeq Z_{u_1} \oplus Z_{u_2} \oplus ...  \oplus Z_{u_k} \oplus Z^l$, где $u_1 | u_2 | ... | u_k$, $u_1 > 1$, $l \in N$ (возможно нулевое).

\textbf{Доказательство}

Пусть $a_1, ..., a_n$ "--- порождает $A$, $G$ "--- САГ порядка $n$ с базисом $e_1, ..., e_n$. Тогда существует сюръективный гомоморфизм $\phi: G \rightarrow A, \ \phi(e_i) = a_i$. Пусть $H = ker(\phi) \leq G$.

Выберем в $G$ и $H$ согласованные базисы $g_1, ..., g_n$ и $h_1, ..., h_k$, что $h_i = u_ig_i$. По теореме о гомоморфизме и в силу сюръективности $A \simeq G / H$.

$G = \langle g_1 \rangle \oplus ... \oplus \langle g_n \rangle$ по определению базиса. Тогда $H = \langle u_1g_1 \rangle \oplus ... \oplus \langle u_kg_k \rangle$.

$A \simeq G / H \simeq \langle g_1 \rangle / \langle u_1g_1 \rangle \oplus ... \oplus \langle g_k \rangle / \langle u_kg_k \rangle \oplus Z^{n-k} \simeq Z_{u_1} \oplus Z_{u_2} \oplus ...  \oplus Z_{u_k} \oplus Z^l$. Заметим, что все $u_i = 1$ можно не учитывать, так как $Z_1 = \{e\}$

\vspace{10pt}

\underline{Опр.} \textbf{Примарная циклическая группа (соответсвующая простому числу $p$)} "--- $Z_{p^k}$, где $p$ "--- простое.

\vspace{5pt}

\textbf{Утв. (о разложении конечной циклической группы в прямую сумму примарных циклических}

Любая конечная группа раскладывается в прямую сумму примарных циклических

Пусть $n = p_1^{\alpha_1}...p_s^{\alpha_s}$ "--- каноническое разложение на простые множители.
Тогда $Z_n = Z_{p_1^{\alpha_1}} \oplus ... \oplus Z_{p_s^{\alpha_s}}$. 

\textbf{Доказательство}

Пусть $\phi: Z_n \rightarrow \oplus Z_{p_i^{\alpha_i}}$ "--- гомоморфизм в прямую сумму примарных циклических групп (действующий на смежные классы $Z / nZ \simeq Z_n$).

$\phi(a + nZ) = (a + p_1^{\alpha_1}Z, ..., a + p_s^{\alpha_s}Z)$. 

$a + nZ \in ker(\phi) \Leftrightarrow a \vdots p_1^{\alpha_1}...p_s^{\alpha_s} \Leftrightarrow a + nZ = nZ \Leftrightarrow a = 0$.

Значит, ядро тривиально и гомоморфизм инъективен, но $|Z_n| = n = |\oplus Z_{p_i^{\alpha_i}}|$ и гомоморфизм сюръективен, а значит это изоморфизм.

\vspace{10pt}

\textbf{Замечание}

Группы $Z_{p^k}$ и $Z$ неразложимы.

\textbf{Доказательство}

В $Z_{p^k}$ есть единственная подгруппа $H$ порядка $p$ и она циклическая. Любая другая подгруппа в $Z_{p^k}$ примарна и так же содержит $H$, а значит разложения быть не может.

Пусть $Z = H_1 \oplus H_2$ "--- разложение в циклические группы, то есть $H_1 = \langle n \rangle, \ H_2 = \langle m \rangle$. Тогда $\langle nm \rangle \subset H_1 \cap H_2$ и пересечение нетривиально

\vspace{5pt}

\textbf{Замечание}

Доказанные теоремы дополняет теорема о единственности разложения: конечные группы в разложении определены однозначно, степень $l$ так же определена однозначно.

Разложение конечной конечнопорожденной абелевой группы единственно с точностью до порядка примарных групп в разложении.
\end{document}