\documentclass{article}

\usepackage[a4paper]{geometry}
\usepackage{mathtools,amssymb}
\usepackage{array}
\newcolumntype{P}[1]{>{\centering\arraybackslash}p{#1}}

\usepackage[T1,T2A]{fontenc}
\usepackage[utf8]{inputenc}
\usepackage[russian]{babel}
\usepackage{multirow}
\usepackage{hhline}
\usepackage{cancel}

\usepackage[useregional]{datetime2}
\usepackage[thinlines]{easytable}
\usepackage{cancel}

\title{Теория групп. Лекция 13}
\author{Штепин Вадим Владимирович}
\date{\DTMdate{2019-11-28}}

\begin{document}
\maketitle

\textbf{Теорема (вторая теорема Силова)}

Пусть $|G| = p^nm$, $m$ не делится на $p$, $P$ "--- силовская $p$-подгруппа в $G$. Тогда $\forall p$-подгруппы $Q \leq G \  \exists x \in G \  Q \leq P^x$. То есть, любая $p$-подгруппа в $G$ вкладывается в силовскую $p$-подгруппу (группа, сопряженная силовской сама силовская).

\vspace{5pt}

\textbf{Доказательство}

Рассмотрим действие $Q$ на $G/P$ "--- множество левых смежных классов. 

$Q(aP) = QaP$ "--- действие левыми сдвигами. Пусть $\Omega = G/P$, $|\Omega| = \frac{|G|}{|P|} = m$. По формуле орбит:

$m = |\Omega| = |Q(a_{i_1})P| + ... + |Q(a_{i_s})P|$, где $s$ "--- количество орбит действия.

$|Q(a_{i_j})P| = \frac{|Q|}{|St(a_{i_j}|}$ по формуле размера орбиты. Так как $|Q|$ "--- степень $p$, то $|Q(a_{i_j})P|$ "--- тоже степень $p$. Если в формуле орбит все слагаемые делятся на $p$, то и $m$ делится на $p$, а это противоречие.

Значит, $\exists$ орбита $Q(a_{i_j})P$ мощности 1. Для нее верно $Qa_{i_j}P = a_{i_j}P \Rightarrow Qa_{i_j}Pa_{i_j}^{-1} = a_{i_j}Pa_{i_j}^{-1} = P^x$, где $x = a_{i_j}$. $QP^x = P^x \Leftrightarrow Q \subset P^x$, $|P^x| = |P|$, так как сопряжение "--- это автоморфизм.

\vspace{10pt}

\textbf{Следствие}

Все силовские $p$-подгруппы сопряжены между собой и $N_p = |G:N_G(P)|$ (число силовских $p$-подгрупп), где $P$ "--- произвольная силовская $p$-подгруппа.

\vspace{5pt}

\textbf{Доказательство}

Пусть $Q = P$ во второй теореме Силова, тогда $P = P_1^x$ "--- некоторая силовская $p$-подгруппа. Значит, все силовские $p$-подгруппы сопряжены некоторой подгруппе. $N_p = \frac{|G|}{|N_G(P)|}$, если рассмотреть $N_p$ как размер орбиты $P$ при действии сопряжениями, $N_G(P)$ "--- нормализатор $P$ (т.е. стабилизатор при действии сопряжениями)

\vspace{10pt}

\textbf{Следствие}

$|G| = p^nm$, $p$ "--- простое и $m$ не делится на $p$. Тогда $m \ \vdots \ |N_p|$.

\vspace{5pt}

\textbf{Доказательство}

$N_p = \frac{|G|}{|N_G(P)|}$, причем $P \leq N_G(P)$, так как $\forall a \in P \  aP = Pa$. Значит, $|N_G(P)| = t|P|$, $t$ "--- натуральное число. Тогда $N_p = \frac{p^nm}{p^nt} = \frac{m}{t}$ и $m$ делится на $N_p$.

\vspace{10pt}

\textbf{Пример}

Всякая группа порядка 15 абелева

\textbf{Доказательство}

$|G| = 3*5 \Rightarrow$ в $G$ есть силовские 3- и 5-подгруппы, причем $N_3 \equiv 1 \  (mod \  3)$ и $5 \ \vdots \ 5$. Значит, $N_3 = 1$. Аналогично получаем, что $N_5 = 1$. Больше силовских подгрупп в $G$ нет. Значит, существует не более двух элементов порядка 3 (элементы силовской 3-подгруппы без нейтрального), не более четырех элементов порядка 5 (элементы силовской 5-подгруппы без нейтрального), и ровно один элемент порядка 1 (нейтральный). По теореме Лагранжа порядки всех элементов "--- это делители порядка группы. Значит, в $G$ есть элементы порядка 15 (и их минимум 8), а группа $G$ циклическая.

\vspace{10pt}

\textbf{Замечание}

В общем случае для $pq$-групп (групп из $pq$ элементов, где $p, q$ "--- простые) утверждать абелевость нельзя, но можно доказать разрешимость.

\vspace{5pt}

\textbf{Теорема}

Пусть $p, q$ "--- различные простые числа, $p < q$ и $|G| = pq$. Тогда $G \simeq C_q \rtimes C_p$. Следовательно, $G$ разрешима.

\vspace{5pt}

\textbf{Доказательство}

$\exists$ силовская $p$-подгруппа $P$ и $q$-подгруппа $Q$ в $G$. Так как порядки групп $P$ и $Q$ "--- простые числа, то $P \simeq C_p, \  Q \simeq C_q$. Покажем, что $Q \triangleleft G$.

$N_q \equiv 1 \  (mod \  q)$ и $p \ \vdots \ N_q$, причем $p < q$ и простое. Значит, $N_q = 1$ и $\forall x \in G \  Q^x = Q$ и $Q \triangleleft G$ по определению.

$P \cap Q = \{e\}$, так как $P \cap Q \leq P$ и $P \cap Q \leq Q$. По теореме Лагранжа, $|P \cap Q| = 1$.

По теореме о произведении нормальной подгруппы на подгруппу $PQ = QP \leq G$, причем $|PQ| = \frac{|P||Q|}{|P \cap Q|} = |G|$ и $PQ = G$. По теореме о разложении группы в полупрямое произведение получаем $G \simeq C_q \rtimes C_p$. Тогда, $G/C_q \simeq C_p$ "--- абелева и разрешима, $C_q$ так же абелева и разрешима. По критерию разрешимости в терминах нормальной подгруппы, $G$ разрешима.

\vspace{10pt}

Если $P \triangleleft G$, то по теореме о разложении в прямое произведение верно $G \simeq P \times Q$, а значит $G$ циклична и абелева (это верно в случае, если $N_p = 1$). Причем верна эквивалентность $G$ абелева $\Leftrightarrow N_p = 1$.

\vspace{10pt}

\textbf{Замечание}

Если $N_p > 1$, то $N_p = q = 1 + \alpha p, \alpha \geq 1$ и $q - 1 \ \vdots \ p$

\vspace{10pt}

\textbf{Пример неабелевой $pq$-группы.}

Пусть $G = \{ A = \begin{vmatrix}
	a & b \\
	0 & 1 \\
\end{vmatrix} \mid det(A) \neq 0\, b \in Z_q, \  a \in Z_q* = Z_q \setminus \{ 0 \}\}$.

$|Z_q*| = q-1 \ \vdots \ p$ "--- абелева, и, по первой теореме Силова в $Z_q*$ найдется силовская $p$-подгруппа, причем она обязательно циклическая и порождается элементом порядка $p$. Пусть это подгруппа $H = \langle a \rangle$. Положим теперь $G = \{ A = \begin{vmatrix}
	a & b \\
	0 & 1 \\
\end{vmatrix}
\mid det(A) \neq 0\, b \in Z_q, \  a \in Z_* = H\}$, и $G$ "--- неабелева.

\vspace{10pt}

\textbf{Теорема (о разложении группы в прямое произведение силовских подгрупп)}

Пусть $|G| = p_1^{k_1}...p_s^{k_s}$ "--- попарно различные простые числа.

Тогда
\begin{enumerate}
	\item силовская $p$-подгруппа нормальна в $G \Leftrightarrow N_p = 1$
	\item $G = P_1 \times ... \times P_s \Leftrightarrow \forall i \ P_i \triangleleft G$
\end{enumerate}

\textbf{Доказательство}

\begin{enumerate}
	\item $P$ "--- силовская $p$-подгруппа, $P \triangleleft G \Leftrightarrow \forall x \in G \ P^x = P \Leftrightarrow N_p = 1$
	\item \textbf{Достаточность}. Пусть $P_1, ..., P_s$ "--- силовские $p_i$-подгруппы и $P_i \triangleleft G$.
	
	Индукция по $s$.
	
	База: Для $s = 1$ очевидно.
	Переход: Пусть теорема верна для всех $G$, что $|G| = p_1^{k_1}...p_{s-1}^{k_{s-1}}$. Рассмотрим произведение первых $s-1$ групп: $P_1...P_{s-1} \triangleleft G, \ P_s \triangleleft G$, причем $P_s \cap (P_1...P_{s-1}) = \{e\}$, так как $\forall z \in P_s \cap (P_1...P_{s-1}) \ ord(z) = 1 \Rightarrow z = e$. Значит, $G = (P_1...P_{s-1}) \times P_s$ по теореме о разложении группы в прямую сумму. К $P_1...P_{s-1}$ применимо предположение индукции, и значит $G = P_1 \times...\times P_s$
	
	\textbf{Необходимость}. Пусть $G = P_1 \times ... \times P_s$. $|P_1 \times ... \times P_s| = p_1^{k_1}...p_s^{k_s} \Rightarrow$ слева присутствуют все силовские подгруппы в $G$. По определению прямого произведения, $\forall i \ P_i \triangleleft G$.
\end{enumerate}

\vspace{10pt}

\textbf{Упражнение}

Пусть $G = A \times B$ "--- внутреннее прямое произведение подгрупп ($A \cap B = \{e\}$ по определению внутреннего прямого произведения). Доказать, что всякая силовская $p$-подгруппа в $G$ раскладывается в прямое произведение силовских $p$-подгрупп в $A$ и $B$

\vspace{5pt}

\textbf{Идея доказательства}

$\forall x \in G \ X = x_Ax_B$, $x_A \in A, \ x_B \in B$ и разложение единственно. Рассмотрим гомоморфизмы $\phi: G \rightarrow A$, $\phi(x) = x_A$ и $\psi: G \rightarrow B$, $\psi(x) = x_B$. Если $P$ "--- силовская $p$-подгруппа, то $\phi(P) \leq A$ и $\psi(P) \leq B$. Очевидно, что $P \leq \phi(P) \times \psi(P)$. Осталось доказать равенство $P = \phi(P) \times \psi(P)$.
\end{document}