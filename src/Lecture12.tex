\documentclass{article}

\usepackage[a4paper]{geometry}
\usepackage{mathtools,amssymb}
\usepackage{array}
\newcolumntype{P}[1]{>{\centering\arraybackslash}p{#1}}

\usepackage[T1,T2A]{fontenc}
\usepackage[utf8]{inputenc}
\usepackage[russian]{babel}
\usepackage{multirow}
\usepackage{hhline}
\usepackage{cancel}

\usepackage[useregional]{datetime2}
\usepackage[thinlines]{easytable}
\usepackage{cancel}

\title{Теория групп. Лекция 12}
\author{Штепин Вадим Владимирович}
\date{\DTMdate{2019-11-21}}

\begin{document}
\maketitle

\section{Теоремы Силова}

Пусть $G$ "--- конечная группа, $|G| = p^nm$, где $p$ "--- простое и $m$ не делится на $p$.

\textbf{Замечание}

В общем случае нельзя говорить о том, что для любого делителя размера группы есть подгруппа такого размера.

\vspace{5pt}

\textbf{Пример}

В $A_4$ нет подгруппы порядка $6$

\vspace{5pt}

\underline{Опр.} В группе $G$ порядка $p^nm$, $p$ "--- простое и $m$ не делится на $p$ подгруппа $H$ порядка $p^n$ "--- \textbf{силовская $p$-подгруппа}

\vspace{10pt}

\textbf{Лемма}

Пусть $q = p^n$ и $\Omega_q$ "--- множество всех $q$-элементных подмножеств в $G$. Тогда $|\Omega_q| = C^{p^n}_{p^nm} \equiv m \  (mod \  p)$. В частности, $|\Omega_q|$ не делится на $p$

\textbf{Доказательство}

В $(1 + x)^{p^nm}$ коэффициентом при $x^{p^n}$ является $C^{p^n}_{p^nm}$ "--- искомое число подмножеств. Будем раскрывать этот бином над $Z_p$. Было доказано, что $(a + b)^p \equiv a^p + b^p \ (mod \  p)$

$(1 + x)^{p^nm} = ((1 + x)^p)^{p^{n-1}m} = (1 + x^p)^{p^{n-1}m} = ... = (1 + x^{p^n})^m = 1 + mx^{p^n} + ...$. Коэффициент при $x^{p^n}$ "--- это $m$. Значит, $|\Omega_q| \equiv m \ (mod \  p)$.

\vspace{10pt}

\textbf{Теорема (первая теорема Силова)}

Пусть $G$ "--- конечная группа, $p$ "--- простое, $m$ не делится на $p$ и $q = p^n$. Тогда $G$ имеет силовскую $p$-подгруппу.

\textbf{Доказательство}

Рассмотрим действие $I: G \rightarrow S(\Omega_q) \  I(a): S \rightarrow aS$. По формуле орбит: $\Omega_q = G(S_1) \cup G(S_2) \cup ... \cup G(S_l)$ "--- объединение попарно непересекающихся орбит.

$|\Omega_q| = \sum \limits_{i = 1}^{l_1} |G(S_i)|$. Если $\forall i \  |G(S_i)|$ делится на $p$, то $|\Omega_q|$ тоже делится на $p$, а это невозможно по лемме. Значит, $\exists S: |G(S)|$ не делится на $p$. Пусть $St(S)$ "--- стабилизатор подмножества $S$. 

По теореме о мощности орбиты, $|G(S)| = \frac{|G|}{|St(S)|} = \frac{p^nm}{|St(S)|} = k$, причем $k$ не делится на $p$.

$|St(S)| = \frac{p^nm}{k} \in N$, а значит $m$ делится на $k$, так как $k$ не делится на $p$.

С другой стороны, $\forall g \in St(S)$ верно $gS \subset S \Rightarrow St(S)S \subset S \Rightarrow \forall s \in S \  St(S)s \subset S$. Так как левые сдвиги "--- инъективное отображение, то $|St(S)| = |St(S)s| \leq |S| = p^n$. Но, так как $m$ делится на $k$, то $|St(S)| \geq p^n$, а значит $|St(S)| = p^n$ и это силовская $p$-подгруппа.

\vspace{10pt}

\textbf{Замечание}
\begin{enumerate}
	\item $|G(s) = \frac{p^nm}{|St(S)|} = m$ "--- орбита, содержащая элемент, стабилизатор которого "--- силовская $p$-подгруппа.
	\item $St(S)s = S$
\end{enumerate}

\vspace{10pt}

\textbf{Теорема (третья теорема Силова)}

$|G| = p^nm$, $m$ не делится на $p$. Пусть $N_p$ "--- число силовских $p$-подгрупп в $G$. Тогда $N_p \equiv 1 \ (mod \  p)$.

\textbf{Доказательство}

$\Omega_q$ "--- множество $q$-элементных подмножеств в $G$, $q = p^n$, $I: G \rightarrow S(\Omega_q)$, $I_a(S) = aS$. Разобьем орбиты на два типа: мощность которых делится на $p$ (первый тип) и мощность которых не делится(второй тип).

Будет доказано, что каждая орбита второго типа содержит (как элемент) единственную силовскую $p$-подгруппу, а каждая орбита первого типа нет.

$St(S)s = S$, где $S$ "--- множество из орбиты второго типа.

$s^{-1}St(S)s = s^{-1}S \in G(S)$. Поскольку сопряжение не меняет мощности, то $s^{-1}St(S)s$ "--- силовская подгруппа.

Покажем, что в орбите второго типа нет двух различных силовских $p$-подгрупп. Пусть не так, значит $P_1, P_2 \in G(S)$ "--- силовские $p$-подгруппы. Значит, $\exists a \in G \  P_1 = aP_2$. Так как $P_1, P_2$ "--- подгруппы, то $e \in aP_2 (= P_1)$ и $e \in P_2$. Значит, $aP_2 \cup P_2 \neq \emptyset$. По свойствам левых смежных классов, $P_2 = aP_2 = P_1$.

По замечанию, $|G(S)| = m$ "--- мощность орбит второго типа. 

Осталось доказать, что орбиты первого типа силовских $p$-подгрупп не содержат.

Пусть $P$ "--- силовская $p$-подгруппа, $G(P)$ "--- орбита первого типа, содержащая $P$, то есть $|G(P)|\ \vdots \  p$, а $St(P)$ "--- стабилизатор $P$.

$|G(P)| = \frac{|G|}{|St(P)|} = \frac{p^nm}{p^n} = m$ "--- противоречие.

По определению стабилизатора, $\forall p \in P \  St(P)p \subset P \Rightarrow |St(P)| \leq |P|$, но, очевидно, что $P$ стабилизирует саму себя, а значит $P \subset St(P) \Rightarrow P = St(P)$.

$|\Omega_q| = \sum |G(S)|$ "--- сумма мощностей орбит первого типа и орбит второго типа, и $|\Omega_q| \equiv m \  (mod \  p)$. Так как мощности орбит первого типа делятся на $p$, то $m|N_p| \equiv m \ (mod \  p)$ и $|N_p| \equiv 1 \  (mod \  p)$, так как силовских подгрупп столько же, сколько орбит второго типа.
\end{document}