\documentclass{article}

\usepackage[a4paper]{geometry}
\usepackage{mathtools,amssymb}
\usepackage{array}
\newcolumntype{P}[1]{>{\centering\arraybackslash}p{#1}}

\usepackage[T1,T2A]{fontenc}
\usepackage[utf8]{inputenc}
\usepackage[russian]{babel}
\usepackage{multirow}
\usepackage{hhline}

\usepackage[useregional]{datetime2}
\usepackage[thinlines]{easytable}
\usepackage{cancel}

\title{Теория групп. Лекция 10}
\author{Штепин Вадим Владимирович}
\date{\DTMdate{2019-11-07}}

\begin{document}
\maketitle

\section{Простота знакопеременных групп $A_n$ при $n > 4$}

\textbf{Теорема}

$A_5$ простая.

\textbf{Доказательство}

Рассмотрим классы сопряженных элементов в $S_5$, состоящие из четных подстановок и проверим, какие из них совпадают с классами сопряженных элементов $A5$, а какие распадаются на два класса:

\renewcommand{\baselinestretch}{1.2} 
\begin{table}
	\centering
	\setlength{\tabcolsep}{10pt}
	\renewcommand{\arraystretch}{1.5}
    \begin{tabular}{|P{0.4cm}|P{2.4cm}|P{2.4cm}|P{2.4cm}|P{2.4cm}|}
    \hline
    & {$|x^{S_5}|$} & $x \in S_5$ & $x \in A_5$ & $|x^{A_5}|$ \\ \hline
    1. & 1 & $e$ & $e$ & $1$ \\
	\hline
    2. & 15 & $(1 2)(3 4)$ & $(1 2)(3 4)$ & $15$  \\
	\hline
    3. & 20 & $(1 2 3)$ & $(1 2 3)$ & $20$ \\
	\hline
    \multirow{2}{*}{4.} & \multirow{2}{*}{24} & \multirow{2}{*}{$(1 2 3 4 5)$} & $(1 2 3 4 5)$ & $12$ \\
	\hhline{~~~--}		 &						&	  								& $(1 2 3 5 4)$ & $12$ \\
	\hline
    \end{tabular}
\end{table}

$(1 2) \in C_{S_5}((1 2)(3 4)) \neq C_{A_5}((1 2)(3 4)) \Rightarrow ((1 2)(3 4))^{S_5} = ((1 2)(3 4))^{A_5}$ по лемме.

Аналогично, $(4 5) \in C_{S_5}((1 2 3)) \neq C_{A_5}((1 2 3)) \Rightarrow ((1 2 3))^{S_5} = ((1 2 3))^{A_5}$

Вычислим $C_{S_5}((1 2 3 4 5))$:

Пусть $\sigma \in C_{S_5}((1 2 3 4 5)) \Leftrightarrow \sigma(1 2 3 4 5) = (1 2 3 4 5)\sigma$

$\sigma = \begin{pmatrix}
1 & 2 & 3 & 4 & 5 \\
\sigma_1 & \sigma_2 & \sigma_3 & \sigma_4 & \sigma_5
\end{pmatrix}$

Значит, $\sigma_1 \rightarrow \sigma_2 \rightarrow \sigma_3 \rightarrow \sigma_4 \rightarrow \sigma_5 \rightarrow \sigma_1$ в цикле $(1 2 3 4 5)$, причем $sigma_{i+1}$ восстанавливается по $\sigma_i$ однозначно, следовательно $C_{S_5}((1 2 3 4 5)) = \langle (1 2 3 4 5) \rangle = C_{A_5}((1 2 3 4 5))$. По лемме, $|((1 2 3 4 5))^{A_5}| = \frac{|((1 2 3 4 5))^{S_5}|}{2}$, а по замечанию, так как  $S_5 = A_5 \cup (4 5)A_5$ (так как $(4 5) \notin A_5$, то $((1 2 3 4 5))^{S_5} = ((1 2 3 4 5))^{A_5} \cup (((1 2 3 4 5))^{(4 5)})^{S_5}$ и $((1 2 3 4 5))^{(4 5)} = (1 2 3 5 4)$

Значит, этот класс сопряженных элементов распадается на два. В таблице показаны представители и мощности классов.

Пусть $\{e\} \neq H \triangleleft A_5$. Согласно одному из критериев, $H$ представляет собой дизъюнктное объединение классов сопряженных элементов, в по теореме Лагранжа, мощность $H$ "--- это делитель мощности $A_5$. Пусть $\delta_i$ "--- индикатор того, что $i$-тый класс вложен в $H$. Тогда $|H| = \delta_1 + 15\delta_2 + 20\delta_3 + 12\delta_4 + 12\delta_5$. Единственный возможный набор $\delta_i \in \{0, 1\}$, подходящий всем условиям, это $\delta_i = 1$ для всех $i$. Значит $A_5$ "--- простая.

\vspace{10pt}

\textbf{Лемма}

$A_n$ порождается тройными циклами  

\textbf{Доказательство}

$\sigma \in A_n \Rightarrow \sigma$ разлагается в произведение четного числа транспозиций. Разобьем их на пары соседних и перемножим: $(i j)(s t) = (i j s)(i s t)$, если транспозиции не пересекаются и $(i j)(j k) = (i j k)$, то есть $\sigma$ разложилось в произведение тройных циклов

\vspace{10pt}

\textbf{Теорема (Д-во Штепина)}

Группы $A_n$ при $n \geq 5$ простые.

\textbf{Доказательство}

Индукция по $n$.

База: $A_5$ простая (доказано)

Переход: Пусть $A_5, ..., A_{n-1}$ простые и $n \geq 6$. Покажем, что $A_n$ простая.

Пусть $H \triangleleft A_n$, $H \neq \{e\}$.

\underline{Идея}: $\exists \pi \in H$, $\pi \neq \{e\}$ и $\pi$ имеет неподвижную точку.

Пусть $\tau \in H$ "--- не имеет неподвижных точек. Б.о.о. $\tau(1) = 2$. Так как $\tau \neq (1 2)$, то $\exists i \  \tau(i) = j$, $i, j \notin \{1, 2\}$. Пусть $k, l \notin \{1, 2, i, j\}, \  k \neq l$. В силу нормальности $H$: $\tau^{(j k l)} \in H$. $\tau^{(j k l)} = (j l k)\tau(j k l) = \sigma$. Причем, $\sigma(1) = 2, \  \sigma(i) = l$. Пусть $\pi= \tau^{-1}\sigma$. $\pi(2) = 2$ и $\pi(j) = l$. Значит, $\pi$ отлично от $e$ и имеет неподвижную точку. Б.о.о. неподвижной точкой будет являться $n$ (иначе перенумеруем элементы). $\pi \in H \cap A_{n-1} = H_1$. $A_{n-1} \leq A_n$ "--- группа элементов, сохраняющих $n$.

$H \triangleleft A_n \Rightarrow H_1 \triangleleft A_{n-1}$. По предположению индукции, $H_1 = A_{n-1} \Rightarrow (1 2 3)H_1 \subset H \Rightarrow (1 2 3)^{A_n} \subset H$, так как $H$ "--- нормальная. Причем, $(1 2 30^{A_n} = (1 2 3)^{S_n}$ так как $(4 5) \in C_{S_n}(1 2 3)$, но $(4 5) \notin A_n$. Значит, $H$ содержит все тройные циклы и $H = A_n$, так как $A_n$ порождается тройными циклами.

\vspace{10pt}

\textbf{Теорема. (Альтернативное д-во)} Группа $A_n$, при $n \ge 5$ - простая
 
\textbf{Доказательство.}
Индукция по $n$. База $n=5.\ A_5$ - простая. Переход: пусть $A_5,\ldots,A_{n-1}$ простые, докажем, что $A_n$ простая. Пусть $\{e\}\neq H \triangleleft A_n$ докажем, что найдется $\pi\in H$,такая что $\pi\neq e$ и $\pi(n)= n$. Возьмем $\tau\neq e$ если $\tau(n)=n$, то $\pi=\tau$, иначе пусть $\tau(n)=a$. Возьмем $i\notin\{n,a\}$ такое, что $\tau(i)\neq i$ (такое найдется т.к иначе $\tau=(an)\notin A_n$). Возьмем различные $k,l\notin\{\tau(i),i,n,a\}$. В силу нормальности $H,\ \tau^{(ikl)}\in H$. \\Пусть
$\pi=\tau^{-1}\tau^{(ikl)}=\tau^{-1}(ikl)\tau(ilk)\in H,\ \pi(n)=n$.
Проверим, что $\pi\neq e$. $\pi(k)=i\neq k$, следовательно $\pi\neq e$.\\
$\pi\in H\cap A_{n-1} = K$. Так как $A_{n-1}\le A_n$ и $H \triangleleft A_n$, то $K \triangleleft A_{n-1}$. Следовательно из предположения индукции $K=A_{n-1} \Rightarrow (123)\in K \le H \Rightarrow (123)^{A_n} \subset H$, так как $H$ нормальная. Причем $(123)^{A_n}=(123)^{S_n}$, т.к $(45)\in C_{S_n}((123))$, но $(45)\notin A_n$. Значит, $H$ содержит все тройные циклы, следовательно $H = A_n$, т.к $A_n$ порождается всеми тройными циклами.

\vspace{10pt}

\textbf{Пример}

Проективные специальные группы $PSL_n(F)$, где $F$ "--- конечное поле.

$PSL_n(F) = SL_n(F) / Z(SL_n(F))$, $Z(SL_n(F)) = \{\lambda E \mid \lambda \in F\}$ "--- центр группы.

\vspace{5pt}

\textbf{Утв. (б/д)}

$PSL_n(F)$ проста, если $|F| \geq 4$ или $n \geq 3$.

\textbf{Теорема}

Группа $SO_3$ над $R$ (группа ортогональных преобразований в $R^3$) простая.

\textbf{Доказательство}

Было доказано, что если $A \in SO_n$, то $A$ приводится к каноническому виду блочно-диагональной матрицы с блоками в виде чисел $\pm 1$ и матриц поворота.

В группе $SO_3$ канонический вид "--- это диагональная матрица с $\pm 1$ на главной диагонали, либо блок "--- поворот на угол $\phi$ и блок из числа 1. Общий вид:

$A = \begin{pmatrix}
1 & 0 & 0 \\
0 & cos(\phi) & -sin(\phi) \\
0 & sin(\phi) & cos(\phi)
\end{pmatrix}$

(Можно перенумеровать вектора базиса, чтобы блок был снизу)

Всякая такая матрица соответствует повороту вокруг какой-то прямой на какой-то угол. Пусть $H \triangleleft SO_3$ и $H \neq \{E\}$. Заметим, что сопряжение в $SO_3$ это отображение $A \rightarrow B^{-1}AB$ "--- переход к новому базису с матрицей перехода $B$.

Значит, $A^B$ "--- поворот на угол $\phi$ вокруг прямой с направляющим вектором $B^{-1}(e_1)$, так как $A^B(B^{-1}(e_1)) = B^{-1}(e_1)$. Если $B$ пробегает все возможные матрицы из $SO_3$, то $A_B$ пробегает все возможные повороты на угол $\phi$ вокруг всех осей, проходящих через начало координат.

\underline{Идея}: покажем, что в $H$ содержатся все повороты на угол $\gamma \in U(0)$ "--- некоторая окрестность точки 0. Тогда в $H$ лежат вообще все повороты (так как композиция поворотов "--- это поворот на сумму углов).

Пусть $B_{\psi}$ "--- поворот вокруг $e_2$ на угол $\psi \in [0, \frac{\pi}{2}$ и $A \neq E$. 

Тогда: $[A, B_{\psi}] \in H \  \forall B_{\psi} \in SO_3$.

$B_{\frac{\pi}{2}} = \begin{pmatrix}
0 & 0 & -1
0 & 1 & 0
1 & 0 & 0
\end{pmatrix}$

$[A, B_{\frac{\pi}{2}]} = \begin{pmatrix}
cos(\phi) & sin(\phi) & 0 \\
-cos(\phi)sin(\phi) & cos^2(\phi) & -sin(\phi) \\
-sin^2(\phi) & sin(\phi)cos(\phi) & cos(\phi)
\end{pmatrix}$

$tr([A, B_{\frac{\pi}{2}}]) = 2cos(\phi) + cos^2(\phi) < 3$, так как $\phi \neq 0$ ($A \neq E$). Значит, $[A, B_{\frac{\pi}{2}}]$ "--- поворот на ненулевой угол $\gamma_0$

Отображение $\psi \rightarrow [A, B_{\psi}]$ непрерывно, и угол поворота $\gamma$ матрицы $[A, B_{\psi}]$ так же непрерывно изменяется, причем при $\psi = 0$ имеем $\gamma = 0$, а при $\psi = \frac{\pi}{2} \ :  \gamma = \gamma_0$. По теореме о промежуточных значениях непрерывной функции, в $H$ есть матрицы поворота на все углы промежутка $[0, \gamma_0]$, а значит и вообще все углы. Следовательно, $H = SO_3$

\end{document}
