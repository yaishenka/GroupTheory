\documentclass{article}

\usepackage[a4paper]{geometry}
\usepackage{mathtools,amssymb}

\usepackage[T1,T2A]{fontenc}
\usepackage[utf8]{inputenc}
\usepackage[russian]{babel}

\usepackage[useregional]{datetime2}
\usepackage{cancel}

\title{Теория групп. Лекция 8}
\author{Штепин Вадим Владимирович}
\date{\DTMdate{2019-10-24}}

\begin{document}
\maketitle

\section{Основная теорема о коммутанте}

\underline{Опр.} \textbf{Взаимный коммутант} $[H, K] = \langle [h, k] \mid h \in H, k \in K \rangle$, $H,K \leq G$.

\vspace{5pt}

\textbf{Замечание.}
Если $H, K \triangleleft G$, то $[H, K] \triangleleft G$.

\textbf{Доказательство}

$\forall h \in H, k \in K, g \in G \  g[h, k]g^{-1} = [ghg^{-1}, gkg^{-1}] \in [H, K]$ в силу нормальности $H$ и $K$.

\vspace{10pt}

\textbf{Утв. (критерий нормальности)}

$H \triangleleft G \Leftrightarrow [H, G] \subset H$

\textbf{Доказательство}
\begin{enumerate}
	\item Пусть $H \triangleleft G$. Тогда $[h, x] = hxh^{-1}x^{-1} \in H$, так как $xh^{-1}x^{-1} \in H$. Значит, $[H, G] \subset H$
	\item Пусть $[H, G] \in H$. Рассмотрим $a \in H, x \in G$. $a^x = x^{-1}ax = aa^{-1}x^{-1}ax = a[a^{-1}x^{-1}] \in H$. По критерию нормальности, $H \triangleleft G$
\end{enumerate}

\vspace{10pt}

\textbf{Теорема (основная теорема о коммутанте)}

Пусть $G$ "--- группа, $G'$ "--- коммутант. Тогда:
\begin{enumerate}
	\item $G/G'$ "--- абелева
	\item $H \triangleleft G, G/H$ "--- абелева $\Leftrightarrow G' \leq H \leq G$ 
\end{enumerate}

\textbf{Доказательство}
\begin{enumerate}
	\item $G' \triangleleft \  G$ по следствию. Пусть $p: G \rightarrow G/G'$ "--- каноническая сюръекция. $\forall x, y \in G \  [p(x), p(y)] = p([x, y])$, так как $p$ "--- гомоморфизм. $[x, y] \in G' = Ker(p) \Rightarrow p([x, y]) = e \Rightarrow p(x)p(y) = p(y)p(x)$. По сюръективности: $\forall a, b \in G/G' \exists x, y \in G' \  p(x) = a, p(y) = b$ и $ab = ba$.
	
	\item Пусть $H \triangleleft G$ и $G/H$ абелева. $\forall x, y \in G$ верно $xHyH = yHxH \Leftrightarrow xyH = yxH \Leftrightarrow [x, y]H = H \Leftrightarrow [x,y] \in H$ и $G' \subset H$.
	
	Пусть $G' \leq H \leq G$.
	
	$[H, G] \subset G' \leq H \Rightarrow H \triangleleft G$
	
	$G' \leq H \Rightarrow \forall x, y \in G \  [x, y] \in H \Rightarrow [x, y]H = H \Rightarrow xyH = yxH  \Rightarrow xHyH = yHxH$ и $G/H$ "--- абелева.
\end{enumerate}

\vspace{10pt}

\textbf{Следствие}

Коммутант "--- наименьшая нормальная подгруппа в $G$, что факторгруппа по ней абелева. Коммутант можно рассматривать как меру неабелевости. Чем больше в группе коммутаторов, отличных от $e$, тем больше она отлична от абелевой.

\section{Нормальная подгруппа, порожденная множеством}

\textbf{Напоминание.} Если $M \subset G$, то $\langle M \rangle = \cap_{H \leq G, M \subset H} H$

\vspace{5pt}

\underline{Опр.} \textbf{Нормальная подгруппа, порожденная множеством $M$} "--- это группа $\langle M \rangle_{n} = \cap_{H \triangleleft G, M \subset H} H$

\vspace{5pt}

\textbf{Замечание.}
$M \subset \langle M \rangle_n \triangleleft G$

\textbf{Теорема (об описании нормальной подгруппы, порожденной множеством)}

$\langle M \rangle_n = \langle m^x \mid m \in M, x \in G \rangle$

\textbf{Доказательство}

Обозначим $\langle M^G \rangle = \langle m^x \mid m \in M, x \in G \rangle$.

Покажем, что $\langle M^G \rangle \subset \langle M \rangle_n$.

$\forall m \in M \  m \in \langle M \rangle_n \Leftrightarrow \forall x \in G m^x \in \langle M \rangle_n$. Значит, $\langle M^G \rangle \subset \langle M \rangle_n$.

Проверим, что $\langle M^G \rangle \triangleleft G$. Пусть $x \in G$. $\langle M^G \rangle^x = \langle m^{yx} \mid m \in M, y \in G \rangle \subset \langle M^G \rangle \Rightarrow \langle M^G \rangle \triangleleft G \Rightarrow \langle M^G \rangle = \langle M \rangle_n$

\vspace{10pt}

\textbf{Теорема}

Пусть $G = \langle M \rangle$. Тогда $G' = \langle [m_1, m_2] \mid m_1, m_2 \in M \rangle_n$

\textbf{Доказательство}

Обозначим $H = \langle [m_1, m_2] \mid m_1, m_2 \in M \rangle_n \subset G'$.

$\forall m_1, m_2 \in M \  [m_1, m_2]^x \in G'$, так как $[m_1, m_2] \in G' \triangleleft G$. Значит, $H \leq G'$ (по предыдущей теореме).

Пусть $p: G \rightarrow G/H$ "--- каноническая сюръекция. $G/H = \langle p(M) \rangle$. $\forall m_1, m_2 \in M \  [p(m_1), p(m_2)] = p([m_1, m_2]) = e$, так как $[m_1, m_2] \in H = Ker(p)$. Значит, $G/H$ порождена взаимно коммутирующими элементами и $G/H$ абелева. По основной теореме о коммутанте $G' \leq H$.

\vspace{5pt}

\textbf{Пример.}

$S_n' = A_n$

\textbf{Доказательство}

$S_n = \langle (i, j) \mid 1 \leq i < j \leq n \rangle$.

Значит, $S_n' = \langle [(i, j), (k, l)] \mid 1 \leq i < j \leq n, 1 \leq k < l \leq n \rangle_n$. Если $i, j, k, l$ не содержат общих элементов, то $[(i, j), (k, l)] = e$. Так же, $[(i, j),(i, k)] = (i, j)(i, k)(i, j)(i, k) = (i, j, k)$

$S'_n = \langle (i, j, k) \rangle_n = \langle (i, j, k) \rangle$, так как сопряженный к циклу длины 3 "--- это цикл длины 3. Любую четную подстановку можно разложить в четное число транспозиций и $(i, j)(k, l) = (i, j, k)(j, k, l)$, если транспозиции не пересекаются. Значит, $A_n = \langle (i, j, k) \rangle = A_n$.

\section{Разрешимые и нильпотентные подгруппы}
Обозначим $G^{(0)} = G, G^{(1)} = G' = [G, G], ... , G^{(k)} = [G^{(k-1)}, G^{(k-1)}]$.

\textbf{Замечание.}

$\forall n \  G^{(n)} \triangleleft G$

\textbf{Доказательство}
$G' \triangleleft G$ "--- доказано.
Пусть $G^{(n-1)} \triangleleft G$. Тогда $G^{(n)} = [G^{(n-1)}, G^{(n-1)}] \triangleleft G$.

\vspace{10pt}

\underline{Опр.} $G = G^{(0)} \triangleright G^{(1)}\triangleright G^{(2)} ... $ "--- \textbf{производный ряд} группы $G$.

\underline{Опр.} Группа $G$ \textbf{разрешима}, если $\exists n \in N \  G^{(n)} = \{e\}$, то есть ряд обрывается на единичной подгруппе.

\underline{Опр.} $G_0 = G, G_1 = G' = [G, G_0], ... , G_k = [G, G_{k-1}], ... $.

$G_0 \triangleright G_1 \triangleright G_2 \triangleright ... $ "--- \textbf{нижний центральный ряд}.


\underline{Опр.} Группа $G$ \textbf{нильпотентна}, если $\exists n \in N \  G_n = \{e\}$, то есть нижний центральный ряд обрывается на единичной подгруппе.

\vspace{10pt}

\underline{Опр.} Наименьшее $n$ в определениях "--- ступень разрешимости (нильпотентности).

\vspace{5pt}

\textbf{Утв.}

Всякая нильпотентная группа разрешима.

\textbf{Доказательство}

Покажем, что $\forall n \in N \  G^{(n)} \subset G_n$.

База индукции: $G_1 = G'$.

Переход индукции: Если $G^{(n-1)} \subset G_{n-1}$, то $G^{(n)} = [G^{(n-1)}, G^{(n-1)}] \leq [G, G^{(n-1)}] \subset [G, G_{n-1}] = G_n$

Если группа нильпотентна, то $G_n = \{e\}$ для некоторого $n$ и $G^{(n)} \leq \{e\}$. Значит, $G^{(n)} = \{e\}$.

\vspace{10pt}

\textbf{Примеры}
\begin{enumerate}
	\item Разрешимая группа ступени 1 "--- абелева.
	
	\underline{Опр.} Разрешимая группа степени 2 "--- \textbf{метабелева}.
	
	\item $D_n$ "--- метабелева. Напомним, что $D_n$ "--- это группа вращений правильного $n$-угольника (группа диэдра). $D_n = \langle S, R \rangle$, где $S$ "--- симметрия относительно оси, проходящей через центр многоугольника и некоторую его вершину, а $R$ "--- поворот на угол $\frac{2\pi}{n}$. Можно показать, что $SRS^{-1} = R^{n-1}$. Тогда $[S, R] \subset \langle R \rangle \Rightarrow D_n' \subset \langle R \rangle$ "--- абелева, так как $D_n'' = \{e\}$ 
\end{enumerate}

\end{document}